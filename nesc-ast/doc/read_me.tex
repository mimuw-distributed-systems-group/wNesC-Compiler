\documentclass{article}
\usepackage[polish]{babel}
\usepackage[utf8]{inputenc}
\usepackage[T1]{fontenc}
\usepackage{fancyhdr}
\usepackage{amsmath}
\usepackage{amsfonts}
\usepackage{tikz}
\usepackage{verbatim}
\begin{document}

\section {Generator abstrakcyjnego drzewa syntaksy}
Witaj w krótkim wprowadzeniu do generator abstrakcyjnego drzewa syntaksy w Pythonie. Aktualna wersja generatora jest
kompatybilna z Pythonem 3.x. 

\section{Pisanie własnej klasy AST}

Każda klasa, dla której ma być wygenerowany kod, musi dziedziczyć z klasy BasicASTNode, jeśli jest węzłem w drzewie, lub
BasicASTEnum jeśl ma reprezentować wyliczeniowy typ danych. 

\subsection{Krótki przykład}
Oto fragment pliku generującego kod:

\begin{verbatim}
from ast_core import *
...

class Stmt(BasicASTNode):
    line = IntField()
    column = IntField()

class Empty(BasicASTNode):
    superclass = Stmt

class BStmt(BasicASTNode):
    superclass = Stmt
    block = ReferenceField("Block")

...

if __name__ == "__main__":
    generate_code(DST_LANGUAGE.JAVA, "ast")
\end{verbatim}

\paragraph
Zaczynamy od zaimportowania wszystkiego z biblioteki. Później następują kolejne definicje
klas węzłów drzewa. Plik ''Latte\_bnf.py'' z którego pochodzi ten przykład jest zakończony warunkiem, który sprawia, że
kod klas opisanych wewnątrz zostanie automatycznie wygenerowany jeśli wywołana zostanie komenda linii poleceń
''python3 Latte\_bnf.py''.

\subsection{Specjalne pola}
Aktualnie jedynym specjalnym polem jest pole o nazwie superclass. Pole to pozwala na określenie hierarhii
dziedziczenia. Aktualnie wspierane jest tylko dziedziczenie z jednej klasy. Ustawienie listy jako wartości pola
superclass jest błędem.

\subsection{Generowanie klas drzewa z wnętrza interpretera}
By wygenerować pliki po opisaniu wszystkich klas, które mają występować w naszym drzewie (abstrakcyjnych i nieabstrakcyjnych),
możemy za pomocą wywołania procedury ''generate\_code(lang, dir)'' wygenerować kod. Obowiązkowy parametr \emph{lang}
musi być jednym elementów DST\_LANGUAGES (enuma opisującego języki dla których możliwość generowania kodu jest zaimplementowana w naszej
 bibliotece). Drugi parametr jest opcjonalny i reprezentuje katalog w którym maja zostać zapisane wygenerowane pliki. Domyślną wartością jest aktualny
 katalog roboczy.

\section{Udostępniane metody i klasy}
--TODO--


\end{document}

